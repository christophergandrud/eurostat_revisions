\documentclass[]{article}
\usepackage{fullpage}
\usepackage[authoryear]{natbib}
\usepackage{setspace}
    \doublespacing
\usepackage{hyperref}
\hypersetup{
    colorlinks,
    citecolor=black,
    filecolor=black,
    linkcolor=cyan,
    urlcolor=cyan
}
\usepackage{amssymb,amsmath}
\usepackage{bm}
\usepackage{dcolumn}
\usepackage{booktabs}
\usepackage{url}
\usepackage{tikz}
\usepackage{todonotes}
\usepackage[utf8]{inputenc}
\usepackage{graphicx}
\usepackage{longtable}
\usepackage{todonotes}
\usepackage{lscape}
\usepackage{float}

\usepackage[margins]{trackchanges}


\title{Fiscal Rule Stretching in Europe During Financial Market Stress and Crises}

\author{Christopher Gandrud \\ \emph{City University London} \\ \emph{Hertie School of Governance}\footnote{Please contact Christopher Gandrud
(\href{mailto:christopher.gandrud@city.ac.uk}{\nolinkurl{christopher.gandrud@city.ac.uk}}) or Mark Hallerberg (\href{mailto:hallerberg@hertie-school.org}{\nolinkurl{hallerberg@hertie-school.org}}). Full data and replication material can be found at: \url{https://github.com/christophergandrud/eurostat_revisions}.}
\and
Mark Hallerberg \\ \emph{Hertie School of Governance}}

\begin{document}

\maketitle

\addeditor{MH}
\addeditor{CG}

\begin{center}
    \textbf{Very early, incomplete working draft for\\ Texas A \& M University: Fiscal Policy in Europe Workshop (10-11 December 2015) \\
    Comments and suggestions very welcome.}
\end{center}

\begin{abstract}
    Elected governments have incentives to stretch accounting rules by classifying loss-making and/or indebted endeavors, such as public industries and pension schemes, as off of the public balance sheet. Doing so improves the appearance of the incumbent government to cost-conscious voters and potentially fiscally important international institutions. We expect rule stretching to be especially prevalent during periods of financial market stress and crises given the large expense of responding to these events. In addition, policy options available to aid troubled financial institutions, such as liquidity assistance, bad banks, and bank nationalizations, are often hard to classify as being inside or outside of the government sector, thus making it more likely that governments will stretch how they are classified. To test these propositions, we examine revisions to government debt and deficit figures made by the European statistical agency--Eurostat. These revisions frequently occur because this politically independent agency re-classifies member state created organisations as being within the government sector, when a national government had originally classified them as outside. We find that debt figures are more likely to be revised upwards for years close to national elections, especially when these are endogenous elections. Such election effects are strengthened further by financial market stress. Our research underlines the importance of having a vigilant and politically independent government statistical agency during periods of financial market stress to ensure reliable government finance statistics.
\end{abstract}

\section{Introduction}

This paper examines why governments stretch the rules when they determine how policies impact their debts and deficits, especially during periods of financial market stress and turmoil. By ``rule stretching'' we mean that if the fiscal implications of a policy are potentially ambiguous, that a decision is made to minimise its debt and/or debt implications. This research question affects other literatures in political science in interesting ways. One strand examines the relationship between the reporting of data and the quality of governance \cite[e.g.][]{Hollyer2014}. \cite{Alt2014} explore the relationship among the transparency of fiscal data, elections, and pressure from the European Union. They argue that EU member states were more likely to violate the statistical agency's rules for reporting budget data when fiscal transparency was low and when it was an election year in ways that made the government look better to voters. A related strand focuses on the economic vote and whether or not voters notice or incorporate information about these revisions into their decisions. \cite{KayserLeininger2015} find that the press, and voters as a result, do not pay attention to revised figures.

Our paper builds on the insights of Kayser and Leininger on revisions and of Alt, Lassen, and Wehner on elections and fiscal transparency to explore the implications of their arguments for government action. If voters care most about reported, rather than actual, figures then governments have an incentive to manipulate them before elections. One would expect greater manipulations in ways that make the government's stewardship of the public balance sheet look better as elections approach. We can measure these activities by looking at revisions of budget figures, and in particular debt. We also explore three alternative hypotheses. The first alternative focuses on ``shocks'' to the economy, which in turn affect government statistics. Countries with more shocks may engage in more revisions, with the bias depending upon the nature of the shocks (positive or negative). One can expect that the shocks are symmetrical, where the greater the shock the more the potential revision, or that one direction requires more revisions, with  the relative elasticities of taxes making one direction more volatile than another. This would be the standard ``economic'' argument. These revisions, however, should be do to ``economic'' effects and not from conscious government manipulation. The forecasting literature in economics includes such measures. The second argument centers on the role of international institutions. Governments may report figures in ways to please an international actor, which in turn can damage the standing of the government in the eyes of its voters. The final one considers institutional arguments about who generates the numbers. In some countries, the government does the reporting, while in others an independent body plays this role.  Of course, these three arguments may be related in theoretically interesting ways. For example, a country experiencing a positive shock may be more likely to call an election in countries where elections are endogenous. [In this version of the paper we examine only the first, with data for the others coming.]

Our dataset is composed only of European Union member states. This is useful for several reasons. They face one international organisation that has two sets of rules on economic matters for its members that are often closely related, namely those rules for Member States inside the eurozone and those outside, which are nevertheless the same when it comes to the reporting of the data. Moreover, there is one body that can provide comparative data that we can analyse. While fiscal figures are initially generated by national agencies, as part of Stability and Growth Pact (SGP) deficit and debt limit enforcement, all Member States face one set of accounting rules (the European System of Accounts or ESA) for data they report to the EU and they have one body, Eurostat, that ultimately enforces it. Eurostat regularly revises Member State data so that it conforms to the common rules. This means that differences across countries in initial reporting are not due to different accounting rules, but instead different interpretations of these rules. Finally, they are all parliamentary or semi-presidential systems where parliamentary elections are crucial for the formation of governments.

We find that in Europe debt figures are more likely to be revised upwards for years close to national elections, especially when these are endogenous elections. This suggests that governments are more prone to rule stretch closer to elections. Such election effects are strengthened further by financial market stress. Our research underlines the importance of having a vigilant and politically independent government statistical agency during periods of financial market stress to ensure reliable government finance statistics.

\section{Political budget cycles and fiscal gimmicks}

The revision of government-reported statistics is of interest for several reasons.

The reporting of statistics, or the lack of it, may tell one about the overall governance of a given country. \cite{Hollyer2014} argue that the lack of reporting is not random. They record the availability of data for 125 countries over a thirty-year period. They analyse this data with Bayesian Item Response model to create a transparency index. They then use this index to predict the quality of governance in autocracies. In other work, they anticipate that this type of transparency affects government accountability, as well as collective action (\cite{hollyerforthcoming}). \footnote{\cite{jervin2013} tells a similar story for African countries. The quality of data is generally good when governments have state capacity to provide them. When those governments get into trouble, however, so that their capacity declines the quality of the data also drops.}

While \cite{Hollyer2014} consider how voters might react under conditions when they get data and when they do not, there is a separate literature on the economic vote that assumes that voters have no trouble accessing regular data on the state of the economy and they use this to decide about the competence of the government in managing economic policy. One debate considers whether voters are prospective or retrospective. On the latter, the assumption is that voters observe outcomes, and they decide whether to support the incumbent government. In response to the strong assumptions about voter knowledge, there is a growing literature on ``real-time fiscal policy''. \cite{KayserLeininger2015} argue that the press ignores revisions to data. Moreover, \cite{kayser_peress} find that voters pay particular attention to data they do not experience themselves. They observe unemployment, for example, but they do not observe economic growth and rely on reported figures when making decisions about which party to support. In our case, we are interested in government fiscal data that only the government can report. If one combines the insights in the two papers, they suggest that voters make decisions on the first set of data they see. This would make manipulation of such data potentially rewarding for governments that seek to remain in office.

We build on this insight for our core argument. Following \cite{nordhaus1975} as well as \cite{Alt2014}, one would anticipate that there are opportunistic business cycles where governments manipulate macro-economic tools at their disposal in an effort to make the economy look better before an election. \cite{clark2003} finds the tools a government uses depends upon the logic of the Mundell-Fleming model: if one assumes that capital is mobile, a government relies on monetary policy when the exchange rate is flexible while it leans on fiscal policy when the exchange rate is fixed. Both Nordhaus and Clark, however, focus on actual economic output from the use of these instruments. We are interested in the perceived figures prior to an election. There is a possible spin-off on the Clark argument. One can expect that extra spending in EU member states with flexible exchange rates has no macro-economic payoffs, while it does have payoffs in countries with fixed exchange rates, which will mostly be those in the eurozone.\footnote{Exceptions would be those countries that fix their exchange rate to the euro or that have very narrow bands. Some central and East European countries like Lithuania, for example, have (or had) currency boards that maintained a de facto fix. Denmark has chosen not to join the eurozone, but it maintains a tight band around the euro for its kronor.} Governments may, however, want to look good to their voters on these figures even in the flexible exchange rate countries in the dataset. This suggests that there should be cycles in revisions to figures according to elections, but that the revisions should be in all member states. \cite{DeCastro2013} in fact find using pre-crisis data that fiscal data for years closer to elections are revised by Eurostat more.

Moreover, we add two additional parts to the model. First, and separate from \cite{Alt2014} who focus on all election years, we anticipate that manipulation of figures is greater when the government calls the election instead of allowing it to go to full term. The reason follows the insights of \cite{clark2003} on the use of instruments, but with the expectation that elections called quickly--largely for non-fiscal or economic reasons--do not give the government time to affect the economy directly either through monetary or fiscal policy. \cite{Kayser2005} finds that manipulation of the economy before endogenous elections is too cumbersome and too difficult to time. This suggests that the manipulations are substitutes for actual instrument use.

Second, we are interested in circumstances under which governments more easily can manipulate the figures. Where are the ``grey'' areas in terms of the classification of assets and liabilities? For \cite{Alt2014}, those ``grey'' areas are in the budgetary shadows, in places where there is less transparency. We focus on specific types of operations in this paper. In particular, Eurostat has been active clarifying initially ambiguous rules has been on how to treat operations that involve the financial sector \cite[see][]{GandrudHallerberg2016}. There is a potential endogeneity problem to using actual operations, however; ultimately, a government decides which tools to use, as \cite{GandrudHallerberg2016} explore in more detail. As we explain below, we measure  overall financial market stress instead of actual realizations, which are part of what the government is manipulating in the first place.

Domestic voters, however, may not be the actors the government is trying to impress with its budget statistics. There is evidence that countries intentionally distort statistics so that they receive some sort of payoff from an international organisation. For example, \cite{kerner2016} find that some African countries kept their per capita incomes below the eligibility threshold for World Bank’s International Development Association (IDA). This is a clear case where governments change their official statistics in response to an expectation from an international organization.

Reported outcomes in terms of economic and fiscal data have a real impact on policy especially in European Union countries. GDP figures affect what type of co-financing governments receive from the European Union for a range of activities, such as environmental protection, guarantees for loans that Small and Medium-sized Enterprises (SMEs) take, and the construction of roads.\footnote{The amounts can be substantial as a percentage of total funding for a given project, up to 85 percent of the cost of the project in regions judged to be “”, which are those with a per capita income below 75 percent of the European Union average.} Under the Stability and Growth Pact, all Member States are expected to have budget balances no worse than 3 percent of GDP and debt burden no greater than 60 percent of GDP. The European Commission decides each year whether a member state has an ``excessive deficit'', with those earning this distinction under the Commission subject to an Excessive Deficit Procedure (EDP). Member states must propose corrective methods. The subset that are also eurozone members face potential penalties, such as fines. Member states that do not adjust their performance also could lose their access to structural and cohesion funds, which are the largest part of the European Union budget.

One could ask whether countries that have clear biases concerning future performance have similar biases concerning current performance. A literature on forecasting considers why there are biases in forecasting numbers. Some governments seem to be eternal optimists; they report figures that that always appear to be better than turn out to be. Others are chronic pessimists, and they have positive ``surprises''. Looking at European Union member states, work from the early years of the euro finds that \citep{hallerbergstrauch2002}. In more recent work, \cite{hallerbergstrauch209} find that European Union member states that fit most closely a ``fiscal contracts'' approach to budgeting have more conservative forecasts than member states that fit better a ``delegation to a strong finance minister'' approach.

This discussion leads us to an institutional argument to explore. \emph{[undeveloped] One could explore these intra-governmental differences as well if space/time/interest. The predictions, however, are not so clean. I can imagine that if you have fiscal contracts--say you are the Netherlands--you might fudge numbers so the you more or less comply with internal targets. If it turns out later you missed a bit, you probably won’t bring down the government. At the same time, precisely because of there is a temptation to do this you might have especially competent staff do the numbers.}

\section{Rule stretching during financial crises}

Governments in countries that experience financial market stress and crises face considerable fiscal difficulties \cite[see][]{Laeven2012} that heighten politicians' incentives to rule stretch. At the same time, and an unexplored area in the literature, the policy options available to politicians to respond to financial crises--e.g. buying equity in a failing bank, bad banks, and bank nationalisations--present rule stretching opportunities. As such, we expect politicians to engage in rule stretching even more during periods of financial market stress and crises.

\cite{GandrudHallerberg2016} argue that many of the policies available to politicians, especially during the 2008-2011 crisis in Europe were both rarely used previously and had potentially ambiguous debt and deficit implications. For example, if a government buys equity in a troubled bank as, for example, Ireland did with Irish Nationwide Building Society (INBS) and the Educational Building Society (EBS) in 2010 has the government spent money or does it have a different, but equivalently valued asset? Does the transaction increase the deficit, or is it what ESA 95--the version of the ESA in force at the onset of the crisis--termed a ``financial transaction'', with no effect on the deficit? If a government owns a stake in an AMC, as the German government did with its AMCs, are the AMC's liabilities government liabilities counted against its debt? Or are they what the ESA 95 terms ``contingent liabilities'', where the government is liable only if the AMC cannot repay them?

Given governments' strong incentives to minimise debt and deficit increases around elections and especially during crises, they have strong incentives to initially classify such policies in the best possible light, e.g. as not increasing the government debts and deficits. In other words, politicians have very strong incentives and many opportunities to fiscal rule stretch during periods of financial market stress.

\section{Hypotheses}

We test the following hypotheses regarding fiscal rule stretching--as measured by Eurostat revisions to member states' debt and deficit figures--around elections and during financial market stress:

\begin{quote}
    $H_{1}$: Debt revisions will be smaller for years further from national government elections.
\end{quote}

\begin{quote}
    $H_{2}$: Debt revisions will be greater for years when there are endogenous elections.
\end{quote}

\begin{quote}
    $H_{3}$: The effects predicted by $H_{1}$ and $H_{2}$ will be stronger when a country also has high financial market stress.
\end{quote}

\section{Empirical tests: Set up}

To test our hypotheses, we ran a series of linear regressions with cumulative debt and deficit revisions made by Eurostat as the dependent variable and a number of relevant political and economic indicators on the right-hand side.

\subsection{Eurostat revisions}

To test these hypotheses we gathered all of the revisions that Eurostat made to EU member debt and deficit figures from 2003 through 2013.\footnote{PDF files with the fiscal figures were downloaded from \url{http://ec.europa.eu/eurostat/news/news-releases}. Accessed March 2015.} Eurostat publishes revisions bi-annually--typically once at the end of April and again in late October. These revisions cover government finance statistics released within the previous four years. As such, the unit of analysis in the following regressions is each Eurostat revision. For every year that government statistics are revised, there are up to seven revisions: The first revisions occurs in October of the initial reporting year and continue for three years thereafter.

We created a variable of \emph{cumulative revisions for debts and deficits} up to and including each revision point. We used these as our dependent variables. For cumulative debt revisions, the variable ranged from -1.1 and 12.7 percent of GDP. For cumulative deficit revisions, the variable ranged from -4.5 and 1.1 percent of GDP. It is important to note that all of these revisions were not due to updates made to the denominator: GDP. Eurostat reports GDP revisions separately from corrections made to the classification of policies' debt and deficit implications. As \cite{DeCastro2013} found for similar Eurostat data in shorter pre-financial crisis sample, there is a clear tendency for debt statistics to be revised upward and deficit statistics downward, indicating that policies are more expensive than initially reported.

\subsection{Right-hand variables}

To test whether governments are more likely to stretch the rules closer to elections, we use Gandrud's \citeyearpar{gandrudYrcurnt} \emph{years to election} variable. The variable counts down from the year that is possibly the furthest away based on national election requirements. Election years are recorded as zero. Not only do we expect that governments are more likely to use rule stretching as elections approach, but that rule stretching should be more prevalent when governments are not able to choose the when the election occurs so as to present themselves in the best fiscal light to voters. As such, we also include a dummy variable that is one for \emph{required elections} (e.g. non-endogenous elections) and zero for all other years. The variable is from \cite{Brender2008}. It was updated and corrected by Hallerberg and Wehner. We expect that the revisions will be greater for years when there is an endogenous election.

To examine how responding to financial market stress may exacerbate governments' fiscal accounting rule stretching behavior, we included Gandrud and Hallerberg's \citeyearpar{finstress_paper} continuous ``FinStress'' measure of real-time perceptions of financial market stress. This measure is created by analyzing monthly content from Economist Intelligence Unit texts on banking and financial markets using a statistical method called kernel principal component analysis. Unlike previous post-hoc dichotomous measures of financial crisis \citep[e.g. measures compiled by][]{Laeven2012,ReinhartRog2010}, the authors argue this measure captures what we are most interested in when trying to understand policy choices: what stress policy-makers perceived and so responded to at the time. It also does not rely on ad hoc methods of determining when a crisis ended, but instead charts perceived intensity over time. To make this monthly measure comparable with our other variables, we found yearly averages. FinStress is able to vary between zero and one, with higher values indicating more perceived financial market stress. In the sample it varies between 0.12 and 0.76.

Because we hypothesize that the effect of election timing on rule stretching will increase at higher levels of financial market stress, we will focus on interactions between FinStress and the election timing and election endogeneity variables.

So far we have only operationalised the domestic--voter--audience for rule stretching. To examine the ``international'' audience, we examine whether the level of government gross debts and deficits effects the propensity of rule stretching and therefore revisions by Eurostat. Perhaps countries that have higher debts and deficits and so are in threat of or have actually breached the SGP's limits will be more likely to rule stretch. If this were the case, we would expect that in our sample, higher deficits will be more strongly associated with revisions than debts. This is because the 3 percent of GDP deficit limit was the focus of EDP enforcement much more so than the 60 percent of GDP debt limit until the advent of the eurozone debt crisis. To test this we gathered general government deficits as a percentage of GDP from Eurostat.\footnote{Available at: \url{http://ec.europa.eu/eurostat/}. Accessed December 2015.} For gross debts, we use information reported by the World Bank's Development Indicators.\footnote{Available at: \url{http://data.worldbank.org/data-catalog/world-development-indicators}. Accessed December 2015.} Both sets of numbers were updated by 2015.

In some model specifications we included a dummy variable for eurozone membership. An important consideration in enforcement through the EDP is governments' good faith in returning to compliance. Fiscal rule stretching may be viewed as a bad faith move. As such countries subject to SGP enforcement may be less likely to rule stretch. All member states are covered under the monitoring procedures of the Stability and Growth Pact. However, non-eurozone members, particularly the United Kingdom, face weaker or non-existent enforcement actions if they breach the SGP's debt and deficit limits. As such, we would expect that higher revisions would be associated with being outside of the eurozone. Conversely, inspired by \cite{clark2003}, perhaps countries in the eurozone having fixed exchange rates need to rely more rule stretching and so eurozone membership is positively associated with revisions. Because of these conflicting predictions, we do not have strong priors on eurozone membership's effect.

\section{Empirical Tests: results}

Tables \ref{debt_results} and \ref{deficit_results} show our results from linear regressions with cumulative debt and deficit revisions as the dependent variables, respectively. As Eurostat has more time to examine member state government policies, we would expect on average that the cumulative revisions will grow over the course of the three and a half year period during which they are revised. To account for this we also include a variable counting the years since the original publication year on the right-hand side. All models also include country fixed effects to help account for unobserved country variation.

\subsection{Debt Revisions}

We can see in Table \ref{debt_results}, model 3 that years from an election is estimated to have a negative effect on debt revisions in models where it is not interacted with FinStress. Government debt figures tend to be revised less for years further away from the next scheduled election. Or stated another way, debt revisions are larger for years closer to elections. This finding is similar in direction and magnitude to what \cite{DeCastro2013} previously uncovered with similar data over a shorter time-span. New findings can be seen in subsequent models of Table \ref{debt_results}. In model 5 we can see that debt revisions were larger in endogenous election years, where a government called an election when they were not required to. Conversely, non-endogenous election years see no difference in debt revisions compared to non-election years. Looking at the interactive models 8--10, their findings are in line with our third hypothesis that the two election effects increase for years that also have higher financial market stress as measured by the FinStress variable.

To understand the direction and magnitude of these interactive effects, figures \ref{me_finstress_elect} and \ref{me_finstress_endog_elect} show marginal effects of the election variables at various levels of financial market stress. At high FinStress levels, e.g. above 0.55,\footnote{\cite{finstress_paper} find that countries classified by \cite{Laeven2012} as being in banking crises tend to have FinStress scores above 0.55, especially in developed economies such as those found in the EU.} the marginal effect of being one more year removed from an election on debt revisions is about -0.2 percentage points of GDP. Figure \ref{me_finstress_endog_elect} shows that having an endogenous election during a similar high stress period increases debt revisions by about 2 percentage points of GDP.

To get a sense of how these average effects may play out in each country, we predicted the debt revisions by the third revision year for 27 EU member states.\footnote{We did not include the newest member--Croatia--as its figures have been subject to Eurostat oversight for relatively few years.} Using estimates shown in model 9 of Table \ref{debt_results}, we predicted revisions for non-election years, endogenous election years, and non-endogenous election years at observed country minimum and maximum FinStress values. The predictions are shown in Figure \ref{country_predict_debt_required}. Please note that while we used observed country FinStress ranges these plots make frequent out of sample predictions as most countries did not have endogenous and non-endogenous elections at these levels of stress. Basic country effects correspond to our priors, namely Greece has very large revisions regardless of election year and type. Typically, when countries are fitted to have endogenous elections at high stress levels, e.g. Belgium, Denmark, Latvia, Ireland, Spain, and so on, we predict that their debt figures will be revised by about 3.25 percentage points of GDP.

\subsection{Deficit Revisions}

We also examined if election type and election timing affected the need for revisions to governments' deficit figures (Table \ref{deficit_results}). First, it is interesting to note that while higher debt levels did not appear to be related to debt figures that required more revisions, higher deficit levels are associated with higher deficit revisions. Perhaps this is related to the primary importance of the 3 percent deficit target placed by Stability and Growth Pact monitors for most of the sample until the Eurozone debt crisis.\footnote{Remember that the sample used in our regressions goes through 2011, only one year after the real start of the debt crisis.} Similarly, we find that being in the eurozone decreases the estimated size of deficit, but not debt revisions. Perhaps this is because countries in the eurozone, which are subject to SGP enforcement, were under greater scrutiny and penalty for breaching the SGP's deficit limits. This suggests that having a relatively strong fiscal monitor could decrease governments' tendency to manipulate figures in fixed exchange rate regimes.

Let's move onto the electoral and financial market stress variables. There does not appear to be an interactive relationship between financial market stress and years to election on deficit revisions. This could be because responses to financial market stress--bank nationalizations, guarantees, and so on--predominantly impact government debts rather than deficits and so opportunities for rule stretching are opportunities for debt rule stretching specifically. There does appear to be a statistically significant relationship between non-endogenous elections and FinStress that is similar to the interaction between endogenous elections and stress in the debt revisions models (see Figure \ref{me_finstress_non_endog_deficit} for the marginal effects). More work is needed to understand this finding.

\begin{landscape}
    
% Table created by stargazer v.5.2 by Marek Hlavac, Harvard University. E-mail: hlavac at fas.harvard.edu
% Date and time: Thu, Feb 11, 2016 - 11:58:28
\begin{table}[!htbp] \centering 
  \caption{Linear Regression Estimation of \textbf{Debt} Revisions} 
  \label{debt_results} 
\tiny 
\begin{tabular}{@{\extracolsep{5pt}}lcccccccccc} 
\\[-1.8ex]\hline 
\hline \\[-1.8ex] 
 & \multicolumn{10}{c}{\textit{Dependent variable:}} \\ 
\cline{2-11} 
\\[-1.8ex] & \multicolumn{10}{c}{Cumulative Debt Revisions} \\ 
\\[-1.8ex] & (1) & (2) & (3) & (4) & (5) & (6) & (7) & (8) & (9) & (10)\\ 
\hline \\[-1.8ex] 
 Cum. Revisions (lag) & 0.628$^{***}$ & 0.758$^{***}$ & 0.628$^{***}$ & 0.757$^{***}$ & 0.626$^{***}$ & 0.731$^{***}$ & 0.631$^{***}$ & 0.557$^{***}$ & 0.563$^{***}$ & 0.630$^{***}$ \\ 
  & (0.020) & (0.021) & (0.021) & (0.021) & (0.021) & (0.022) & (0.020) & (0.026) & (0.022) & (0.020) \\ 
  & & & & & & & & & & \\ 
 Election Timing & 0.047$^{*}$ &  & 0.047$^{*}$ &  & $-$0.031 &  &  &  &  &  \\ 
  & (0.026) &  & (0.027) &  & (0.047) &  &  &  &  &  \\ 
  & & & & & & & & & & \\ 
 Unscheduled Elect. &  & 0.227$^{*}$ &  & 0.229$^{*}$ &  & $-$0.560$^{**}$ &  &  &  &  \\ 
  &  & (0.129) &  & (0.132) &  & (0.235) &  &  &  &  \\ 
  & & & & & & & & & & \\ 
 Scheduled Elect. &  & $-$0.003 &  & $-$0.001 &  & $-$0.057 &  &  &  &  \\ 
  &  & (0.070) &  & (0.073) &  & (0.128) &  &  &  &  \\ 
  & & & & & & & & & & \\ 
 Financial Stress &  &  & 0.039 & 0.191 & $-$1.082$^{*}$ & $-$0.049 &  &  &  &  \\ 
  &  &  & (0.327) & (0.257) & (0.643) & (0.278) &  &  &  &  \\ 
  & & & & & & & & & & \\ 
 Eurozone &  &  &  &  &  &  & 0.064 &  &  &  \\ 
  &  &  &  &  &  &  & (0.160) &  &  &  \\ 
  & & & & & & & & & & \\ 
 Cent. Gov. Debt &  &  &  &  &  &  &  & 0.003$^{*}$ &  &  \\ 
  &  &  &  &  &  &  &  & (0.002) &  &  \\ 
  & & & & & & & & & & \\ 
 Gen. Gov. Deficit &  &  &  &  &  &  &  &  & $-$0.011 &  \\ 
  &  &  &  &  &  &  &  &  & (0.009) &  \\ 
  & & & & & & & & & & \\ 
 Fiscal Trans. &  &  &  &  &  &  &  &  &  & 0.003 \\ 
  &  &  &  &  &  &  &  &  &  & (0.003) \\ 
  & & & & & & & & & & \\ 
 Elect. Timing*Fin. Stress &  &  &  &  & 0.506$^{**}$ &  &  &  &  &  \\ 
  &  &  &  &  & (0.250) &  &  &  &  &  \\ 
  & & & & & & & & & & \\ 
 Unscheduled.Elect*Fin. Stress &  &  &  &  &  & 5.818$^{***}$ &  &  &  &  \\ 
  &  &  &  &  &  & (1.443) &  &  &  &  \\ 
  & & & & & & & & & & \\ 
 Scheduled.Elect*Fin. Stress &  &  &  &  &  & 0.384 &  &  &  &  \\ 
  &  &  &  &  &  & (0.731) &  &  &  &  \\ 
  & & & & & & & & & & \\ 
 Constant & 0.654$^{***}$ & 0.387$^{***}$ & 0.648$^{***}$ & 0.361$^{**}$ & 0.831$^{***}$ & 0.353$^{**}$ & 0.676$^{***}$ & 0.541$^{***}$ & 0.816$^{***}$ & 0.615$^{***}$ \\ 
  & (0.178) & (0.144) & (0.188) & (0.151) & (0.208) & (0.152) & (0.235) & (0.194) & (0.158) & (0.219) \\ 
  & & & & & & & & & & \\ 
\hline \\[-1.8ex] 
Country FE? & Yes & Yes & Yes & Yes & Yes & Yes & Yes & Yes & Yes &  \\ 
Observations & 1,553 & 1,236 & 1,494 & 1,189 & 1,494 & 1,189 & 1,553 & 1,230 & 1,495 & 1,553 \\ 
R$^{2}$ & 0.547 & 0.656 & 0.546 & 0.655 & 0.547 & 0.660 & 0.546 & 0.396 & 0.498 & 0.547 \\ 
Adjusted R$^{2}$ & 0.539 & 0.647 & 0.537 & 0.646 & 0.538 & 0.651 & 0.538 & 0.384 & 0.488 & 0.538 \\ 
\hline 
\hline \\[-1.8ex] 
\textit{Note:}  & \multicolumn{10}{r}{$^{*}$p$<$0.1; $^{**}$p$<$0.05; $^{***}$p$<$0.01} \\ 
\end{tabular} 
\end{table} 

\end{landscape}

\begin{landscape}
    
% Table created by stargazer v.5.2 by Marek Hlavac, Harvard University. E-mail: hlavac at fas.harvard.edu
% Date and time: Wed, Feb 10, 2016 - 17:00:43
\begin{table}[!htbp] \centering 
  \caption{Linear Regression Estimation of \textbf{Deficit} Revisions} 
  \label{deficit_results} 
\tiny 
\begin{tabular}{@{\extracolsep{5pt}}lcccccccccc} 
\\[-1.8ex]\hline 
\hline \\[-1.8ex] 
 & \multicolumn{10}{c}{\textit{Dependent variable:}} \\ 
\cline{2-11} 
\\[-1.8ex] & \multicolumn{10}{c}{Cumulative Deficit Revisions} \\ 
\\[-1.8ex] & (1) & (2) & (3) & (4) & (5) & (6) & (7) & (8) & (9) & (10)\\ 
\hline \\[-1.8ex] 
 Yrs. Since Original & 0.684$^{***}$ & 0.586$^{***}$ & 0.635$^{***}$ & 0.684$^{***}$ & 0.818$^{***}$ & 0.682$^{***}$ & 0.818$^{***}$ & 0.681$^{***}$ & 0.819$^{***}$ & 0.806$^{***}$ \\ 
  & (0.019) & (0.024) & (0.020) & (0.019) & (0.017) & (0.019) & (0.017) & (0.019) & (0.017) & (0.018) \\ 
  & & & & & & & & & & \\ 
 Eurozone & $-$0.033 &  &  &  &  &  &  &  &  & $-$0.019 \\ 
  & (0.059) &  &  &  &  &  &  &  &  & (0.043) \\ 
  & & & & & & & & & & \\ 
 Cent. Gov. Debt &  & $-$0.001 &  &  &  &  &  &  &  &  \\ 
  &  & (0.001) &  &  &  &  &  &  &  &  \\ 
  & & & & & & & & & & \\ 
 Gen. Gov. Deficit &  &  & 0.011$^{***}$ &  &  &  &  &  &  & 0.006$^{**}$ \\ 
  &  &  & (0.004) &  &  &  &  &  &  & (0.003) \\ 
  & & & & & & & & & & \\ 
 Election Timing &  &  &  & $-$0.005 &  & $-$0.006 &  & $-$0.021 &  &  \\ 
  &  &  &  & (0.009) &  & (0.010) &  & (0.017) &  &  \\ 
  & & & & & & & & & & \\ 
 Unscheduled Elect. &  &  &  &  & 0.010 &  & 0.009 &  & $-$0.073 & 0.029 \\ 
  &  &  &  &  & (0.042) &  & (0.042) &  & (0.075) & (0.077) \\ 
  & & & & & & & & & & \\ 
 Scheduled Elect. &  &  &  &  & 0.026 &  & 0.021 &  & 0.056 & 0.049 \\ 
  &  &  &  &  & (0.023) &  & (0.023) &  & (0.041) & (0.037) \\ 
  & & & & & & & & & & \\ 
 Financial Stress &  &  &  &  &  & $-$0.069 & $-$0.068 & $-$0.290 & $-$0.059 & 0.035 \\ 
  &  &  &  &  &  & (0.120) & (0.082) & (0.236) & (0.090) & (0.096) \\ 
  & & & & & & & & & & \\ 
 Elect. Timing*Fin. Stress &  &  &  &  &  &  &  & 0.100 &  &  \\ 
  &  &  &  &  &  &  &  & (0.091) &  &  \\ 
  & & & & & & & & & & \\ 
 Unscheduled.Elect*Fin. Stress &  &  &  &  &  &  &  &  & 0.604 & $-$0.357 \\ 
  &  &  &  &  &  &  &  &  & (0.446) & (0.510) \\ 
  & & & & & & & & & & \\ 
 Scheduled.Elect*Fin. Stress &  &  &  &  &  &  &  &  & $-$0.241 & $-$0.213 \\ 
  &  &  &  &  &  &  &  &  & (0.236) & (0.209) \\ 
  & & & & & & & & & & \\ 
 Constant & $-$0.039 & $-$0.034 & $-$0.053 & $-$0.062 & $-$0.065 & $-$0.052 & $-$0.055 & $-$0.016 & $-$0.065 & $-$0.029 \\ 
  & (0.086) & (0.080) & (0.060) & (0.065) & (0.047) & (0.068) & (0.048) & (0.075) & (0.049) & (0.060) \\ 
  & & & & & & & & & & \\ 
\hline \\[-1.8ex] 
Country FE? & Yes & Yes & Yes & Yes & Yes & Yes & Yes & Yes & Yes &  \\ 
Observations & 1,553 & 1,230 & 1,495 & 1,553 & 1,236 & 1,494 & 1,189 & 1,494 & 1,189 & 1,143 \\ 
R$^{2}$ & 0.559 & 0.442 & 0.497 & 0.559 & 0.761 & 0.559 & 0.763 & 0.559 & 0.764 & 0.730 \\ 
Adjusted R$^{2}$ & 0.551 & 0.431 & 0.487 & 0.551 & 0.755 & 0.550 & 0.757 & 0.550 & 0.758 & 0.722 \\ 
\hline 
\hline \\[-1.8ex] 
\textit{Note:}  & \multicolumn{10}{r}{$^{*}$p$<$0.1; $^{**}$p$<$0.05; $^{***}$p$<$0.01} \\ 
\end{tabular} 
\end{table} 

\end{landscape}


\begin{figure}
    \caption{Marginal Effect of Election Timing (years to election) at Various Levels of Financial Market Stress on \textbf{Debt} Revisions}
    \label{me_finstress_elect}

    \begin{center}
        \includegraphics[scale=0.4]{figures/finstress_elect_me.pdf}
    \end{center}

	{\scriptsize{Shaded area represents 95\% confidence interval.}}

\end{figure}

\begin{figure}
    \caption{Marginal Effect of an Endogenous Election at Various Levels of Financial Market Stress on \textbf{Debt} Revisions}
    \label{me_finstress_endog_elect}

    \begin{center}
        \includegraphics[scale=0.4]{figures/finstress_endog_elect_me.pdf}
    \end{center}

	{\scriptsize{Shaded area represents 95\% confidence interval.}}

\end{figure}

\begin{figure}
	\caption{Predicted \textbf{Debt} Revisions in Three Years After Publication for Years with Different Election Types/Non-election Years}
    \label{country_predict_debt_required}
    \begin{center}
    	\includegraphics[scale=0.7]{figures/country_predict_required.pdf}
    \end{center}

	{\scriptsize{High and Low stress values refer to country minimum and maximum FinStress scores in the sample.\\
    Croatia excluded due to a small number of revision years.\\
    Shaded areas show 95\% confidence intervals.
}}

\end{figure}

\begin{figure}
    \caption{Marginal Effect of a Non-Endogenous Election at Various Levels of Financial Market Stress on \textbf{Deficit} Revisions}
    \label{me_finstress_non_endog_deficit}
    \begin{center}
        \includegraphics[scale=0.4]{figures/finstress_non_endog_deficit_me.pdf}
    \end{center}

	{\scriptsize{Shaded area represents 95\% confidence interval.}}

\end{figure}


\section{Conclusion}

In this paper we have attempted to understand government fiscal rule stretching behavior during financial market stress and crises. The European Union provides a unique opportunity for studying this behavior as it has a common set of statistical rules--the ESA--and has an independent monitor that frequently revisits and revises member state balance sheet statistics to ensure that they are similarly in line with these rules. We significantly deepen findings in previous work that elections are associated with revisions to government debt and deficit figures. We specifically examine behavior during endogenous elections and periods of financial market stress. We find that government debt rule stretching is much more prevalent during financial crises and when elections are endogenous, when governments have few opportunities to affect economic fundamentals.

Overall, it appears that deficit rule stretching, at least during our sample, may be more about avoiding running afoul of the Stability and Growth Pact's three percent deficit limit, rather than being a result of electoral incentives or financial market stress. Nonetheless, more work is needed to understand the observed relationship between non-endogenous elections and rule stretching that we observe here.

A major takeaway from our work is that even among a group of developed economies with generally strong economic institutions, that fiscal rule stretching is common and can significantly affect our knowledge about government spending and financial obligations. This is especially true during financial market stress. As such independent government accounting agencies, such as Eurostat, are a crucial component of get the numbers right, even if it takes a few years.


\clearpage

\bibliographystyle{apsr}
\bibliography{main.bib}

\clearpage

\section*{Appendix}

We ran a number of tests to examine whether or not governments select into endogenous elections according to the prevailing level of financial market stress. Table \ref{finstress_endog} shows results from a logistic regression where we tried to predict having an endogenous election in a year for our sample based on annual average FinStress level. We can see that there is a null result. We made a similar finding when running a multinomial logistic regression (not shown), with endogenous and non-endogenous elections as categories and no election as the reference category.


% Table created by stargazer v.5.2 by Marek Hlavac, Harvard University. E-mail: hlavac at fas.harvard.edu
% Date and time: Thu, Dec 03, 2015 - 13:27:54
\begin{table}[!htbp] \centering 
  \caption{Logistic Regression Estimation of Having an Endogenous Election} 
  \label{finstress_endog} 
\begin{tabular}{@{\extracolsep{5pt}}lc} 
\\[-1.8ex]\hline 
\hline \\[-1.8ex] 
 & \multicolumn{1}{c}{\textit{Dependent variable:}} \\ 
\cline{2-2} 
\\[-1.8ex] & Endogenous Election \\ 
\hline \\[-1.8ex] 
 FinStress & $-$0.137 \\ 
  & (0.154) \\ 
  & \\ 
 Constant & 0.195$^{*}$ \\ 
  & (0.111) \\ 
  & \\ 
\hline \\[-1.8ex] 
Country FE? & Yes \\ 
Observations & 216 \\ 
Log Likelihood & 31.788 \\ 
Akaike Inf. Crit. & $-$7.575 \\ 
\hline 
\hline \\[-1.8ex] 
\textit{Note:}  & \multicolumn{1}{r}{$^{*}$p$<$0.1; $^{**}$p$<$0.05; $^{***}$p$<$0.01} \\ 
\end{tabular} 
\end{table} 


\end{document}

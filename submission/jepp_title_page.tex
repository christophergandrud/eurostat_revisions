\documentclass[]{article}
\usepackage{fullpage}
\usepackage[authoryear]{natbib}
\usepackage{setspace}
    \doublespacing
\usepackage{hyperref}
\hypersetup{
    colorlinks,
    citecolor=black,
    filecolor=black,
    linkcolor=cyan,
    urlcolor=cyan
}
\usepackage{amssymb,amsmath}
\usepackage{bm}
\usepackage{dcolumn}
\usepackage{booktabs}
\usepackage{url}
\usepackage{tikz}
\usepackage{todonotes}
\usepackage[utf8]{inputenc}
\usepackage{graphicx}
\usepackage{longtable}
\usepackage{todonotes}
\usepackage{lscape}
\usepackage{float}


\title{Fiscal Rule Stretching in the European Union}

\author{Christopher Gandrud \\ \emph{City University London} \\ \emph{Hertie School of Governance}\footnote{Christopher Gandrud is Lecturer in Quantitative International Political Economy at City University London and Post-doctoral fellow at the Hertie School of Governance. Please contact him at Rhind Building, City University London, EC1V 0HB, London, United Kingdom
(\href{mailto:christopher.gandrud@city.ac.uk}{\nolinkurl{christopher.gandrud@city.ac.uk}}). Mark Hallerberg is Professor of Public Management and Political Economy at the Hertie School of Governance, Friedrichstrasse 180, Berlin 10117, Germany (\href{mailto:hallerberg@hertie-school.org}{\nolinkurl{hallerberg@hertie-school.org}}). Thank you to workshop participants at Texas A \& M University. This work was supported by the Deutsche Forschungsgemeinschaft under grant number HA5996/2-1. Replication material can be found at: \url{https://github.com/christophergandrud/Eurostat_revisions}.}
\and
Mark Hallerberg \\ \emph{Hertie School of Governance}}

\begin{document}

\maketitle

\begin{abstract}
Elected governments have incentives to stretch accounting rules. Doing so improves the government’s appearance to cost-conscious voters and important international institutions, especially in the European Union where there are externally imposed budget limits. We expect rule stretching to be especially prevalent during periods of financial market stress and crisis given these events' large costs and that policy responses are often not obviously classifiable as being inside or outside of the government sector. To test these propositions, we examine debt data revisions made by the European statistical agency--Eurostat. We find that debt figures are more likely to be revised upwards for countries facing European Union budget enforcement and years close to national elections, especially when elections are unscheduled. Financial market stress increases the magnitude of the election effects. Our research underlines the importance of having a vigilant and politically independent government statistical agency to ensure reliable government finance statistics.
\end{abstract}


\textbf{Keywords:} fiscal policy, European Union, financial crisis, electoral budget cycles, Eurostat

\textbf{Word count:} 7,482

\end{document}
